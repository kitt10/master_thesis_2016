%----------------------------------------------------------------------------------------
%	ABSTRACT PAGE
%----------------------------------------------------------------------------------------

\begin{abstract}
\addchaptertocentry{\abstractname} % Add the abstract to the table of contents

Proprioception and tactile sensing in insect-like legged robots is a fast, \mbox{illumination} insensitive and biologically inspired way of ground perception. In this thesis, 14 virtually generated terrains are classified based on the mentioned sensor types for a simulated version of hexapod robot AMOS~II. A feedforward neural network framework equipped with a novel network pruning algorithm has been developed for classification. We observe over $ 92\% $ classification accuracy on deterministic terrain data and $ 72\% $ on manually noised data. The pruning algorithm removes unimportant synapses (generally more than $ 90\% $) from a fully-connected network, while the classification accuracy does not drop significantly. The number of input neurons is reduced by $ 65\% $, resulting in the minimal network structure for the classification problem. A theory of using minimal structures for feature selection is proposed. The thesis outcome consists of a minimal neural network capable of terrain classification based on selected features of proprioceptive and tactile sensory signals.

\end{abstract}