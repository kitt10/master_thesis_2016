\chapter{Terrain Classification for AMOS II}
\label{chapter:05:terrain_classification}

The account of the research should be presented in a manner suitable for the field. It should be complete, systematic, and sufficiently detailed to enable a reader to understand how the data were gathered and how to apply similar methods in another study. Notation and formatting must be consistent throughout the thesis, including units of measure, abbreviations, and the numbering scheme for tables, figures, footnotes, and citations. One or more chapters may consist of material published (or submitted for publication) elsewhere. See “Including Published Material in a Thesis or Dissertation” for details.

chapter intro
1/2 page

\section{Experimental Environment Specification}
target machine description

3-5 pages

\subsection{Hexapod Robot AMOS II}
hardware, hexapod info

\subsection{LPZ Robots Simulation}
simulation info

\section{Virtual Terrains Determination}
parameters + brief analysis
2 pages

\section{Net Input Fixation}
Determination of sensors to be used and its transformation into a feature vector

2-3 pages

\section{Data Acquisition}
Description of how the data has been acquired from the simulation and saved as .txt, adding terrain noise

2 pages
\subsection{Terrain Noise}

\section{Data Processing}
Cleaning the data (deleting incomplete ones), adding signal noise, transformation into datasets, splitting into training-validation-testing sets

2-3 pages
\subsection{Signal Noise}

\section{Training and Classification}

Neural net training with several parameters and comparison with training with scikit-neuralnetwork library

2-3 pages

\subsection{Scikit-neuralnetwork library}
brief description of the library and its usage 1/2 pages

\section{Overall process summary}
diagram of the overall work

1 page