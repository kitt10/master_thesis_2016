\chapter{Method Implementation} \label{app:implementation_details}
In this section, more detailed implementation information, related to the new framework (\cref{app:sec:implementation_of_nn}) and to the terrain classification process (\cref{app:sec:implementation_of_terrain_classification}), is provided.

\section{Implementation of the Neural Network} \label{app:sec:implementation_of_nn}
This section relates to \cref{chap:kitt_nn}. The framework was implemented in programming language Python and named \textit{KITTNN}. A detailed API for the presented classes is attached in \cref{app:sec:api_kitt_nn}.

The following diagram (\ref{img:kitt_nn_package}) shows the structure of the \textit{KITTNN} .py package.

\begin{figure}[H]
  \centering
  \includegraphics[width=0.75\textwidth]{kitt_nn_package.png}
  \caption{kitt\_nn package : Implemented neural network framework}
  \label{img:kitt_nn_package}
\end{figure}

The \textit{KITTNN} implementation is based on some general knowledge gained at school and/or from \citep{online:nn_demystified}, the idea is pretty straight forward.

The overall idea is based on the object-oriented programming. There are three fundamental files containing the main classes corresponding to structural elements - a network, a neuron and a synapse (a connection).

\subsubsection*{kitt\_neuron.py} \label{ssec:kitt_neuron}
The very basic units of a neural net are called neurons. In case of artificial systems, these units are responsible for transfering all their inputs into one output. The behavior is moreless the same for all of the units, therefore a class called \textit{Neuron} implements some basic common functions.

\begin{figure}[H]
  \centering
  \includegraphics[width=0.6\textwidth]{kitt_neuron.png}
  \caption{kitt\_neuron.py : Neuron class inheritance}
  \label{img:kitt_neuron}
\end{figure}

Then, as \cref{img:kitt_neuron} shows, three classes are inherited from the \textit{Neuron} class. Some special functions, like fitting a sample in case of input layers or producing network outcome by output layers respectively, can be implemented this way, while some common functions are shared in the mother \textit{Neuron} class.

\subsubsection*{kitt\_synapse.py} \label{ssec:kitt_synapse}
Next, there is a class representing a neural connection - a synapse. An instance of this class takes care of the corresponding weight and remembers the two connected neurons. 

Additionally, a function called \textbf{remove\_self()} is implemented, which sets the weight to zero and removes the synapse from a database of the corresponding neural net. Then it also checks the two connected neurons, if they have some other connections remaining. If not, they are labeled as \textit{dead}, as they are not a part of the network anymore.

\subsubsection*{kitt\_net.py} \label{ssec:kitt_net}
The network is initialized by creating an instance of \textit{NeuralNetwork()} class from \textit{kitt\_net.py}. The initialization process is illustrated in \cref{img:kitt_net}. Basically, the only parameter is the network structure, which is expected as a \textit{.py iterable} type. 

For instance, a network with 2 input, 5 hidden and 3 output units would be created as \textit{NeuralNetwork(structure=[2, 5, 3])}. Number of hidden layers is not limited.

\begin{figure}[H]
  \centering
  \includegraphics[width=0.7\textwidth]{kitt_net.png}
  \caption{kitt\_net.py : Neural Network Initialization}
  \label{img:kitt_net}
\end{figure}

A learning algorithm is added to the initialized network thereafter (see \cref{sec:learning_algorithm}). The network class implements basic functions like \textit{fit()}, \textit{predict()} in order to be used as a classifier. Moreover, it has some additional utilities like \textit{copy\_self()} or \textit{print\_self()}, which are essential for this work (\cref{sec:gui}, \cref{sec:network_pruning_algorithm}).

\subsection*{Scikit-learn Neural Network Library} \label{ssec:sknn}
In order to verify the functionality of the implemented neural network framework (\textit{KITTNN}), a provided public library is used. As the official description says \citep{misc:sknn}, this library implements multi-layer perceptrons as a wrapper for the powerful \textit{pylearn2} library that is compatible with \textit{scikit-learn} for a more user-friendly and Pythonic interface.

This step has been considered with the aim to test another implementation of the learning algorithm rather than to obtain better classification results. As the only learning parameters are the \textit{net structure}, the \textit{learning rate} and the \textit{number of epochs}, some other default parameters of the tested network are shown in \cref{code:sknn_net}.

\begin{lstlisting}[language=Python, caption={Sknn classifier specification \citep{misc:sknn}}, label=code:sknn_net]
class sknn.mlp.Classifier(layers, warning=None, parameters=None, 
random_state=None, learning_rule=u'sgd', learning_rate=0.01, 
learning_momentum=0.9, normalize=None, regularize=None, 
weight_decay=None, dropout_rate=None, batch_size=1, n_iter=None, 
n_stable=10, f_stable=0.001,  valid_set=None, valid_size=0.0, 
loss_type=None, callback=None, debug=False,  verbose=None)
\end{lstlisting}

\newpage
\section{Implementation of the Terrain Classification} \label{app:sec:implementation_of_terrain_classification}
This section relates to \cref{chap:terrain_classification}. The following \cref{img:app:terrain_classification_process} illustrates the overall terrain classification process and presents an extended version of the diagram in \cref{img:terrain_overall_simple}.

\begin{figure}[H]
  \centering
  \includegraphics[width=1.0\textwidth]{terrain_classification_process}
  \caption{Terrain classification process - overall diagram.}
  \label{img:app:terrain_classification_process}
\end{figure}

In the following sections, the individual steps of the terrain classification process are explained in more detail.

\subsection*{LPZ Robots Simulation} \label{ssec:app:lpzrobots_sim}
This section extends the information from \cref{ssec:amosii_sim}. It is devoted to the simulation of AMOS II using \citep{misc:lpzrobots}

The \textit{lpzrobots} project contains many subprojects. For this study, the most important ones are:

\begin{description}
\item[selforg] : homeokinetic controllers implementation framework
\item[ode\_robots] : a 3D physically correct robot simulator
\end{description}

The project is implemented in \textit{C++} and needs a Unix system to be run. It consists of two main GIT repositories to be forked - lpzrobots and go\_robots. The overall software architecture is shown in \cref{img:lpzrobots_architecture}.

\begin{figure}[H]
  \centering
  \includegraphics[width=0.5\textwidth]{lpzrobots_architecture}
  \caption{Software architecture for LPZRobots and GoRobots. \citep{misc:lpzrobots}}
  \label{img:lpzrobots_architecture}
\end{figure}

To introduce the elements in \cref{img:lpzrobots_architecture}, \textit{ThisSim} is an inherited class of another class called \textit{Simulation} and is initialized everytime the simulation is launched. It integrates all elements together, controls the environment as well as the robot and sets up initial parameters. An instance of the \textit{Agent} class integrates all components of the agent (robot) by using the shown classes.

\subsection*{Terrain Construction in main.cpp} \label{ssec:app:terrain_construction_in_main.cpp}
The \textbf{LpzRobots} AMOS II simulator supports some terrain setting. In the main simulation file (\textit{main.cpp} - see \ref{app:code_documentation}), a \textit{'rough terrain'} substance is being initialized and passed through a handle to a \textit{TerrainGround} constructor.

\begin{lstlisting}[language=C++, caption={Setting a terrain ground in main.cpp}, label=code:terrain_ground]
Substance roughterrainSubstance(terrain_roughness, terrain_slip,
                       terrain_hardness, terrain_elasticity);
oodeHandle.substance = roughterrainSubstance;
TerrainGround* terrainground = new TerrainGround(oodeHandle, 
                       osgHandle.changeColor(terrain_color),
                       "rough1.ppm", "", 20, 25, terrain_height);
\end{lstlisting}

\newpage
\subsection*{Data Storing} \label{ssec:app:data_storing}
It is always recommended to store rough data before some processing, hence the simulator creates \textit{.txt} files of structure symbolized in \cref{txt:rough_data} (with the reference to sensors shortcuts in \cref{tab:proprioceptors}). 

\begin{lstlisting}[language=XML, caption={Rough sensory data files structure}, label=txt:rough_data]
timestep_001;ATRf;ATRm;ATRh;ATLf;...;FRh;FLf;FLm;FLh
timestep_002;ATRf;ATRm;ATRh;ATLf;...;FRh;FLf;FLm;FLh
...
timestep_100;ATRf;ATRm;ATRh;ATLf;...;FRh;FLf;FLm;FLh
\end{lstlisting}

There is a \textit{.txt} file of this structure for every single simulation run in the \textit{root/data/} directory (see \cref{app:code_documentation}).

All the data files are generated by a script called \textit{generate\_txt\_data.py} (\ref{app:code_documentation}). This script takes several arguments, like the number of jobs (simulation runs), terrain types involved or the terrain noise \textit{std} ($ \sigma_p $). Then a loop based on these parameters starts, where the simulation is launched and stopped after ten seconds each iteration. This is performed by calling a bash command (since the simulation is \textit{.cpp} based) and then killing the called process from python. The corresponding \textit{.txt} file is saved after each iteration by the simulation and then copied by the python script to a corresponding folder in \textit{root/data/}.

\begin{figure}[H]
  \centering
  \includegraphics[width=0.7\textwidth]{generating_data}
  \caption{The process of data acquisition from the simulation.}
  \label{img:generating_data}
\end{figure}

In this manner, \textit{.txt} files for all terrains and all mentioned $ \sigma_p $ are saved into a structure illustrated on \cref{img:data_dir_structure}. Each \textit{.txt} file contains approximately 100 lines, one for each simulation step (as shown in \cref{txt:rough_data}). Every line then contains values of the 24 proprioceptive sensors.

\begin{figure}[H]
  \centering
  \includegraphics[width=0.7\textwidth]{data_dir_structure}
  \caption{The structure of rough data directory.}
  \label{img:data_dir_structure}
\end{figure}

Right after the data generation, a script called \textit{clean\_txt\_data.py} (\ref{app:code_documentation}) is used to check the created \textit{.txt} files. As it takes a long time to generate all the data, sometimes the simulation fails and the files are incomplete. Hence the script checks whether there are enough timesteps (at least more than 95) and also if the steps are not messed. Files that fail during the inspection are removed.

\subsection*{Datasets Storing} \label{ssec:app:datasets_storing}
During the overall process description in previous sections, some global process parameters have been collected. These configurations are now passed as arguments to the script called \textit{create\_terrains\_dataset.py} and therefore several datasets of various properties can be generated.

The datasets files are saved in directory \path{root/py/cache/datasets/amos_terrains_sim/} (see \ref{app:code_documentation}). Their structure is based on a powerful serializing and de-serializing Python algorithm implemented under a package called \textit{pickle (cPickle)}. On the same basis a package called \textit{shelve} is used to represent a dataset as a dictionary-link object. The files are saved with the \textit{.ds} suffix.

\subsection*{Trained Network Storing} \label{ssec:app:network_storing}
Finally, the trained network needs a file name, as it is saved the same way as the datasets (see \cref{sec:dataset_creation}) - using the \textit{pickle (cPickle)} package, just with the \textit{.net} suffix.

