\chapter{Code Documentation} \label{app:code_documentation}
A documentation for the \textit{KITTNN} framework implementation is provided in \cref{app:sec:api_kitt_nn}. The API for the implementation of the terrain classification process is in \cref{app:sec:api_scripts}.

\section{Neural Network Framework KITTNN (API)} \label{app:sec:api_kitt_nn}

% NeuralNetwork
\noindent\fcolorbox{lightblue}{lightyellow}{
\begin{minipage}{\textwidth}
  \small {\fontfamily{pcr}\selectfont class kitt\_nn.nn\_structure.kitt\_net.\textbf{NeuralNetwork(structure)}}
\end{minipage}}
\\
\noindent\fcolorbox{lightblue}{lightgray}{
\footnotesize
\begin{minipage}{\textwidth}
  The main class representing an artificial neural network.
\begin{itemize}
\item[\textbf{@}] \textbf{structure} (array-like) : len: number of layers; items: number of neurons per layer
\end{itemize}
\begin{tcolorbox}[boxsep=0pt,top=10pt,left=10pt,right=10pt, bottom=10pt, arc=0pt, auto outer arc, colback=white, colframe=lightgray]
\begin{itemize}
\item[\textbf{def}] \textbf{init} : Creates neurons and makes a fully-connected structure.
\item[\textbf{def}] \textbf{fit} : Feeds the input with samples and trains the model.
\begin{itemize}
\item[@] train\_X    \hspace{15pt}	: array-like, shape (n\_samples, n\_inputs)
\item[@] train\_y    \hspace{18pt}	: array-like, shape (n\_samples, n\_outputs)
\item[@] val\_X      \hspace{25pt}	: array-like, shape (n\_samples, n\_inputs)
\item[@] val\_y      \hspace{28pt}	: array-like, shape (n\_samples, n\_outputs)
\end{itemize}
\item[\textbf{def}] \textbf{predict} : Predicts the output.
\begin{itemize}
\item[@] test\_X    	\hspace{22pt}	: array-like, shape (n\_samples, n\_inputs)  
\item[returns] y\_pred \hspace{23pt}	: array, shape (n\_samples, n\_outputs)
\end{itemize}
\item[\textbf{def}] \textbf{copy} : Creates a copy of self.
\begin{itemize} 
\item[returns] net\_copy \hspace{15pt}	: kitt\_net.NeuralNetwork
\end{itemize}
\end{itemize}
\normalsize
\end{tcolorbox}
\end{minipage}}

% Neuron
\noindent\fcolorbox{lightblue}{lightyellow}{
\begin{minipage}{\textwidth}
  \small {\fontfamily{pcr}\selectfont class kitt\_nn.nn\_structure.kitt\_neuron.\textbf{Neuron(net,layer,id)}}
\end{minipage}}
\footnotesize
\noindent\fcolorbox{lightblue}{lightgray}{
\begin{minipage}{\textwidth}
  The class representing a single neuron unit.
\begin{itemize}
\item[\textbf{@}] \textbf{net} (kitt\_nn.NeuralNetwork) : mother network
\item[\textbf{@}] \textbf{layer} (int) : mother's layer id in the network
\item[\textbf{@}] \textbf{id} (int) : position in the layer
\end{itemize}
\begin{tcolorbox}[boxsep=0pt,top=10pt,left=10pt,right=10pt, bottom=10pt, arc=0pt, auto outer arc, colback=white, colframe=lightgray]
\begin{itemize}
\item[\textbf{def}] \textbf{activate} : Activates the neuron axon by the transfer function.
\item[\textbf{def}] \textbf{set\_bias} : Sets the bias value from the bias matrix hold by the network.
\item[\textbf{def}] \textbf{get\_bias} : Returns current bias value.
\begin{itemize}
\item[returns] b \hspace{23pt}	: float
\end{itemize}
\item[\textbf{def}] \textbf{set\_dead} : Removes the neuron from the net.
\end{itemize}
\normalsize
\end{tcolorbox}
\end{minipage}}

% Synapse
\noindent\fcolorbox{lightblue}{lightyellow}{
\begin{minipage}{\textwidth}
  \small {\fontfamily{pcr}\selectfont class kitt\_nn.nn\_structure.kitt\_synapse.\textbf{Synapse(net,from,to)}}
\end{minipage}}
\footnotesize
\noindent\fcolorbox{lightblue}{lightgray}{
\begin{minipage}{\textwidth}
  The class representing a single synapse.
\begin{itemize}
\item[\textbf{@}] \textbf{net} (kitt\_nn.NeuralNetwork) : mother network
\item[\textbf{@}] \textbf{from} (kitt\_nn.Neuron) : Neuron, where the synapse comes from
\item[\textbf{@}] \textbf{to} (kitt\_nn.Neuron) : Neuron, where the synapse goes to
\end{itemize}
\begin{tcolorbox}[boxsep=0pt,top=10pt,left=10pt,right=10pt, bottom=10pt, arc=0pt, auto outer arc, colback=white, colframe=lightgray]
\begin{itemize}
\item[\textbf{def}] \textbf{set\_weight} : Sets the weight value from the weight matrix hold by the network.
\item[\textbf{def}] \textbf{get\_weight} : Returns current weight value.
\begin{itemize}
\item[returns] w \hspace{23pt}	: float
\end{itemize}
\item[\textbf{def}] \textbf{remove\_self} : Removes the synapse from the net.
\end{itemize}
\normalsize
\end{tcolorbox}
\end{minipage}}

% Neuron
\noindent\fcolorbox{lightblue}{lightyellow}{
\begin{minipage}{\textwidth}
  \small {\fontfamily{pcr}\selectfont class kitt\_nn.nn\_tool.nn\_learning.\textbf{BackPropagation(net)}}
\end{minipage}}
\footnotesize
\noindent\fcolorbox{lightblue}{lightgray}{
\begin{minipage}{\textwidth}
  The class representing the backpropagation learning algorithm.
\begin{itemize}
\item[\textbf{@}] \textbf{net} (kitt\_nn.NeuralNetwork) : mother network
\end{itemize}
\begin{tcolorbox}[boxsep=0pt,top=10pt,left=10pt,right=10pt, bottom=10pt, arc=0pt, auto outer arc, colback=white, colframe=lightgray]
\begin{itemize}
\item[\textbf{def}] \textbf{train} : Trains the mother network.
\begin{itemize}
\item[@] train\_data    \hspace{15pt}	: array-like, shape (n\_samples, 2) [X, y]
\item[@] val\_data    \hspace{25pt}	: array-like, shape (n\_samples, 2) [X, y]
\end{itemize}
\item[\textbf{def}] \textbf{try\_to\_train} : Tries to retrain the mother network.
\begin{itemize}
\item[@] train\_data    \hspace{16pt}	: array-like, shape (n\_samples, 2) [X, y]
\item[@] val\_data    \hspace{25pt}	: array-like, shape (n\_samples, 2) [X, y]
\item[@] req\_accuracy    \hspace{7pt}	: float
\item[returns] retrained \hspace{25pt}	: bool
\end{itemize}
\end{itemize}
\normalsize
\end{tcolorbox}
\end{minipage}}

\section{Terrain Classification Scripts (API)} \label{app:sec:api_scripts}

% generate_txt_data.py
\noindent\fcolorbox{lightblue}{lightblue}{
\begin{minipage}{\textwidth}
  {\fontfamily{pcr}\selectfont \textasciitilde /py/scripts/\textbf{generate\_txt\_data.py}}
\end{minipage}}
\footnotesize
\noindent\fcolorbox{lightblue}{lightgray}{
\begin{minipage}{\textwidth}
  Runs the Amos II simulation and saves sensory data as .txt files.

\begin{tcolorbox}[boxsep=0pt,top=10pt,left=10pt,right=10pt, bottom=10pt, arc=0pt, auto outer arc, colback=white, colframe=lightgray]
\begin{itemize}
\item[\textbf{@}] \textbf{nj} (--n\_jobs) : Number of simulation runs.
\begin{itemize}
\item[-] type    \hspace{15pt}	: int
\item[-] choices \hspace{4pt} 	: [1, 2, ... 1000]
\item[-] default \hspace{4pt} 	: 500 
\end{itemize}
\item[\textbf{@}] \textbf{t} (--terrains) : Terrain (id) to be generated.
\begin{itemize}
\item[-] type    \hspace{15pt}	: int 
\item[-] choices \hspace{4pt} 	: [1, 2, ... 14]
\item[-] default \hspace{4pt} 	: [1, 2, ... 14] 
\end{itemize}
\item[\textbf{@}] \textbf{n} (--noise) : Terrain noise level to be generated.
\begin{itemize}
\item[-] type    \hspace{15pt}	: string  
\item[-] choices \hspace{4pt} 	: ['nn', 'n1p', 'n3p', 'n5p', 'n10p', 'n20p']
\item[-] default \hspace{4pt} 	: 'nn'
\end{itemize}
\item[\textbf{@}] \textbf{nt} (--n\_timesteps) : Number of simulation timesteps.
\begin{itemize}
\item[-] type    \hspace{15pt}	: int 
\item[-] choices \hspace{4pt} 	: [1, 2, ... 1000]
\item[-] default \hspace{4pt} 	: 100 
\end{itemize}
\end{itemize}
\normalsize
\end{tcolorbox}
\end{minipage}}

% clean_txt_data.py
\noindent\fcolorbox{lightblue}{lightblue}{
\begin{minipage}{\textwidth}
  {\fontfamily{pcr}\selectfont \textasciitilde /py/scripts/\textbf{clean\_txt\_data.py}}
\end{minipage}}
\footnotesize
\noindent\fcolorbox{lightblue}{lightgray}{
\begin{minipage}{\textwidth}
  Checks generated .txt files and remove bad/incomplete ones.

\begin{tcolorbox}[boxsep=0pt,top=10pt,left=10pt,right=10pt, bottom=10pt, arc=0pt, auto outer arc, colback=white, colframe=lightgray]
\begin{itemize}
\item[\textbf{@}] \textbf{t} (--terrains) : Terrain (id) to be checked.
\begin{itemize}
\item[-] type    \hspace{15pt}	: int 
\item[-] choices \hspace{4pt} 	: [1, 2, ... 14]
\item[-] default \hspace{4pt} 	: [1, 2, ... 14] 
\end{itemize}
\item[\textbf{@}] \textbf{n} (--noise) : Terrain noise levels to be checked.
\begin{itemize}
\item[-] type    \hspace{15pt}	: string  
\item[-] choices \hspace{4pt} 	: ['nn', 'n1p', 'n3p', 'n5p', 'n10p', 'n20p']
\item[-] default \hspace{4pt} 	: ['nn', 'n1p', 'n3p', 'n5p', 'n10p', 'n20p']
\end{itemize}
\item[\textbf{@}] \textbf{sl} (--sample\_len) : Minimum required sample length.
\begin{itemize}
\item[-] type    \hspace{15pt}	: int 
\item[-] choices \hspace{4pt} 	: [1, 2, ... 1000]
\item[-] default \hspace{4pt} 	: 95 
\end{itemize}
\end{itemize}
\normalsize
\end{tcolorbox}
\end{minipage}}

% create_terrains_dataset.py
\noindent\fcolorbox{lightblue}{lightblue}{
\begin{minipage}{\textwidth}
  {\fontfamily{pcr}\selectfont \textasciitilde /py/scripts/\textbf{create\_terrain\_dataset.py}}
\end{minipage}}
\footnotesize
\noindent\fcolorbox{lightblue}{lightgray}{
\begin{minipage}{\textwidth}
  Creates a dataset out of the cleaned .txt data files and saves it using \textit{cPickle}.

\begin{tcolorbox}[boxsep=0pt,top=10pt,left=10pt,right=10pt, bottom=10pt, arc=0pt, auto outer arc, colback=white, colframe=lightgray]
\begin{itemize}
\item[\textbf{@}] \textbf{rt} (--rem\_terrains) : Terrain (id) to be avoided from the dataset.
\begin{itemize}
\item[-] type    \hspace{15pt}	: int 
\item[-] choices \hspace{4pt} 	: [1, 2, ... 14]
\item[-] default \hspace{4pt} 	: [] 
\end{itemize}
\item[\textbf{@}] \textbf{tn} (--terrain\_noise) : Terrain noise level.
\begin{itemize}
\item[-] type    \hspace{15pt}	: string  
\item[-] choices \hspace{4pt} 	: ['nn', 'n1p', 'n3p', 'n5p', 'n10p', 'n20p']
\item[-] default \hspace{4pt} 	: 'nn'
\end{itemize}
\item[\textbf{@}] \textbf{sn} (--signal\_noise) : Signal noise level.
\begin{itemize}
\item[-] type    \hspace{15pt}	: float 
\item[-] choices \hspace{4pt} 	: [0.0, 0.01, 0.03, 0.05, 0.1]
\item[-] default \hspace{4pt} 	: 0.0
\end{itemize}
\item[\textbf{@}] \textbf{s} (--sensors) : Sensors to be included.
\begin{itemize}
\item[-] type    \hspace{15pt}	: string 
\item[-] choices \hspace{4pt} 	: [atr\_f, atr\_m, ..., fl\_h]
\item[-] default \hspace{4pt} 	: [atr\_f, atr\_m, ..., fl\_h] 
\end{itemize}
\item[\textbf{@}] \textbf{ts} (--timesteps) : Number of timesteps per sensor.
\begin{itemize}
\item[-] type    \hspace{15pt}	: int 
\item[-] choices \hspace{4pt} 	: [1, 2, ... 80]
\item[-] default \hspace{4pt} 	: 40
\end{itemize}
\item[\textbf{@}] \textbf{ds} (--data\_split) : Ratio for train/val/test splitting.
\begin{itemize}
\item[-] type    \hspace{15pt}	: array-like of int ,s.t. sum = 1 
\item[-] choices \hspace{4pt} 	: [0.0, 0.01, ..., 0.99]
\item[-] default \hspace{4pt} 	: [0.7, 0.1, 0.2]
\end{itemize}
\item[\textbf{@}] \textbf{ns} (--n\_samples) : Number of samples per terrain.
\begin{itemize}
\item[-] type    \hspace{15pt}	: int 
\item[-] choices \hspace{4pt} 	: [0, 1, ..., 500]
\item[-] default \hspace{4pt} 	: 500
\end{itemize}
\end{itemize}
\normalsize
\end{tcolorbox}
\end{minipage}}

% kitt_train.py
\noindent\fcolorbox{lightblue}{lightblue}{
\begin{minipage}{\textwidth}
  {\fontfamily{pcr}\selectfont \textasciitilde /py/scripts/\textbf{kitt\_train.py}}
\end{minipage}}
\footnotesize
\noindent\fcolorbox{lightblue}{lightgray}{
\begin{minipage}{\textwidth}
  Trains a \textit{kitt\_nn} neural network on the given dataset and saves it using \textit{cPickle}.

\begin{tcolorbox}[boxsep=0pt,top=10pt,left=10pt,right=10pt, bottom=10pt, arc=0pt, auto outer arc, colback=white, colframe=lightgray]
\begin{itemize}
\item[\textbf{@}] \textbf{ds} (--dataset) : Dataset file name.
\begin{itemize}
\item[-] type    \hspace{15pt}	: string 
\item[-] choices \hspace{4pt} 	: {any string}
\item[-] default \hspace{4pt} 	: '' 
\end{itemize}
\item[\textbf{@}] \textbf{s} (--structure) : Hidden structure of the neural network.
\begin{itemize}
\item[-] type    \hspace{15pt}	: array-like of int
\item[-] choices \hspace{4pt} 	: [1, 2, ..., 1000]
\item[-] default \hspace{4pt} 	: [20] 
\end{itemize}
\item[\textbf{@}] \textbf{lr} (--learning\_rate) : Learning rate for network training.
\begin{itemize}
\item[-] type    \hspace{15pt}	: float
\item[-] choices \hspace{4pt} 	: [0.01, 0.02, ..., 1.0]
\item[-] default \hspace{4pt} 	: 0.03
\end{itemize}
\item[\textbf{@}] \textbf{ni} (--n\_iter) : Number of training epochs.
\begin{itemize}
\item[-] type    \hspace{15pt}	: int
\item[-] choices \hspace{4pt} 	: [1, 2, ..., 1000]
\item[-] default \hspace{4pt} 	: 500 
\end{itemize} 
\end{itemize}
\normalsize
\end{tcolorbox}
\end{minipage}}

% kitt_train.py
\noindent\fcolorbox{lightblue}{lightblue}{
\begin{minipage}{\textwidth}
  {\fontfamily{pcr}\selectfont \textasciitilde /py/scripts/\textbf{kitt\_prune.py}}
\end{minipage}}
\footnotesize
\noindent\fcolorbox{lightblue}{lightgray}{
\begin{minipage}{\textwidth}
  Prunes the given trained network and finds a minimal structure of it with respect to the given dataset.

\begin{tcolorbox}[boxsep=0pt,top=10pt,left=10pt,right=10pt, bottom=10pt, arc=0pt, auto outer arc, colback=white, colframe=lightgray]
\begin{itemize}
\item[\textbf{@}] \textbf{n} (--net) : Trained network file name.
\begin{itemize}
\item[-] type    \hspace{15pt}	: string 
\item[-] choices \hspace{4pt} 	: {any string}
\item[-] default \hspace{4pt} 	: '' 
\end{itemize}
\item[\textbf{@}] \textbf{ra} (--req\_accuracy) : Required classification accuracy of the pruned network.
\begin{itemize}
\item[-] type    \hspace{15pt}	: float
\item[-] choices \hspace{4pt} 	: [0.0, 0.01, 1.0]
\item[-] default \hspace{4pt} 	: [0.99] 
\end{itemize}
\item[\textbf{@}] \textbf{lr} (--learning\_rate) : Learning rate for network re-training.
\begin{itemize}
\item[-] type    \hspace{15pt}	: float
\item[-] choices \hspace{4pt} 	: [0.01, 0.02, ..., 1.0]
\item[-] default \hspace{4pt} 	: 0.03
\end{itemize}
\item[\textbf{@}] \textbf{mi} (--max\_iter) : Maximum number of re-training epochs.
\begin{itemize}
\item[-] type    \hspace{15pt}	: int
\item[-] choices \hspace{4pt} 	: [1, 2, ..., 1000]
\item[-] default \hspace{4pt} 	: 100 
\end{itemize}
\item[\textbf{@}] \textbf{ns} (--n\_stable) : Number of stable epochs for termination.
\begin{itemize}
\item[-] type    \hspace{15pt}	: int
\item[-] choices \hspace{4pt} 	: [1, 2, ..., 1000]
\item[-] default \hspace{4pt} 	: 15 
\end{itemize} 
\end{itemize}
\normalsize
\end{tcolorbox}
\end{minipage}}

\textbf{List of Supplementary Scripts}

\url{create\_mnist\_dataset.py}, \url{create\_xor\_dataset.py}, \url{list\_generated\_datasets.py}, \url{plot\_cl\_results.py}, \url{plot\_feature\_selection.py}, \url{plot\_grid\_search.py}, \url{plot\_nn\_verification.py}, \url{plot\_pa\_results.py}, \url{plot\_sample.py}, \url{plot\_sensor.py}, \url{plot\_transfer\_functions.py}, \url{reed\_prune.py}, \url{set\_and\_store\_params.py}, \url{sknn\_train.py}, \url{terrain\_analysis.py}, \url{zero\_prune.py}