\chapter{Conclussion}
\label{conclussion}

In this section the student must demonstrate his/her mastery of the field and describe the work's overall contribution to the broader discipline in context. A strong conclusion includes the following:

Conclusions regarding the goals or hypotheses presented in the Introduction,
Reflective analysis of the research and its conclusions in light of current knowledge in the field,
Comments on the significance and contribution of the research reported,
Comments on strengths and limitations of the research,
Discussion of any potential applications of the research findings, and
A description of possible future research directions, drawing on the work reported.
A submission's success in addressing the expectations above is appropriately judged by an expert in the relevant discipline. Students should rely on their research supervisors and committee members for guidance. Doctoral students should also take into account the expectations articulated in the University's “Instructions for Preparing the External Examiner's Report”.

2-3 pages

\section{Future Work}
All references:

\citep{article:01:visual} and \citep{article:02:laser} and \citep{article:03:motorsignals} and \citep{article:04:onlinelearning} and \citep{thesis:05:proprioception} and \citep{article:06:haptic} and \citep{thesis:07:proprioception} and \citep{article:08:rhex} and \citep{article:09:roughterrain} and \citep{article:10:pruningalgs} and \citep{book:11:scorpion} and \citep{thesis:12:gaitcontrol}