\chapter{Implementation Details} \label{app:implementation_details}

\section{Terrain Classification Implementation}

\begin{figure}[H]
  \centering
  \includegraphics[width=1.0\textwidth]{terrain_classification_process}
  \caption{Terrain classification process - overall diagram.}
  \label{img:app:terrain_classification_process}
\end{figure}

\subsection*{LPZ Robots Simulation} \label{ssec:app:lpzrobots_sim}
The \textit{lpzrobots} project contains many subprojects. For this study, the most important ones are:

\begin{description}
\item[selforg] : homeokinetic controllers implementation framework
\item[ode\_robots] : a 3D physically correct robot simulator
\end{description}

The project is implemented in \textit{C++} and needs a Unix system to be run. It consists of two main GIT repositories to be forked - lpzrobots and go\_robots. The overall software architecture is shown in \cref{img:lpzrobots_architecture}.

\begin{figure}[H]
  \centering
  \includegraphics[width=0.5\textwidth]{lpzrobots_architecture}
  \caption{Software architecture for LPZRobots and GoRobots. \citep{misc:lpzrobots}}
  \label{img:lpzrobots_architecture}
\end{figure}

To introduce the elements in \cref{img:lpzrobots_architecture}, \textit{ThisSim} is an inherited class of another class called \textit{Simulation} and is initialized everytime the simulation is launched. It integrates all elements together, controls the environment as well as the robot and sets up initial parameters. An instance of the \textit{Agent} class integrates all components of the agent (robot) by using the shown classes.

\subsection*{Terrain Construction in main.cpp} \label{ssec:app:terrain_construction_in_main.cpp}
The \textbf{LpzRobots} AMOS II simulator supports some terrain setting. In the main simulation file (\textit{main.cpp} - see \ref{app:code_documentation}), a \textit{'rough terrain'} substance is being initialized and passed through a handle to a \textit{TerrainGround} constructor.

\begin{lstlisting}[language=C++, caption={Setting a terrain ground in main.cpp}, label=code:terrain_ground]
Substance roughterrainSubstance(terrain_roughness, terrain_slip,
                       terrain_hardness, terrain_elasticity);
oodeHandle.substance = roughterrainSubstance;
TerrainGround* terrainground = new TerrainGround(oodeHandle, 
                       osgHandle.changeColor(terrain_color),
                       "rough1.ppm", "", 20, 25, terrain_height);
\end{lstlisting}

\subsection*{Data Storing} \label{ssec:app:data_storing}
It is always recommended to store rough data before some processing, hence the simulator creates \textit{.txt} files of structure symbolized in \cref{txt:rough_data} (with the reference to sensors shortcuts in \cref{tab:proprioceptors}). 

\begin{lstlisting}[language=XML, caption={Rough sensory data files structure}, label=txt:rough_data]
timestep_001;ATRf;ATRm;ATRh;ATLf;...;FRh;FLf;FLm;FLh
timestep_002;ATRf;ATRm;ATRh;ATLf;...;FRh;FLf;FLm;FLh
...
timestep_100;ATRf;ATRm;ATRh;ATLf;...;FRh;FLf;FLm;FLh
\end{lstlisting}

There is a \textit{.txt} file of this structure for every single simulation run in the \textit{root/data/} directory (see \cref{app:code_documentation}).

All the data files are generated by a script called \textit{generate\_txt\_data.py} (\ref{app:code_documentation}). This script takes several arguments, like the number of jobs (simulation runs), terrain types involved or the terrain noise \textit{std} ($ \sigma_p $). Then a loop based on these parameters starts, where the simulation is launched and stopped after ten seconds each iteration. This is performed by calling a bash command (since the simulation is \textit{.cpp} based) and then killing the called process from python. The corresponding \textit{.txt} file is saved after each iteration by the simulation and then copied by the python script to a corresponding folder in \textit{root/data/}.

\begin{figure}[H]
  \centering
  \includegraphics[width=0.8\textwidth]{generating_data}
  \caption{The process of data acquisition from the simulation.}
  \label{img:generating_data}
\end{figure}

In this manner, \textit{.txt} files for all terrains and all mentioned $ \sigma_p $ are saved into a structure illustrated on \cref{img:data_dir_structure}. Each \textit{.txt} file contains approximately 100 lines, one for each simulation step (as shown in \cref{txt:rough_data}). Every line then contains values of the 24 proprioceptive sensors.

\begin{figure}[H]
  \centering
  \includegraphics[width=0.8\textwidth]{data_dir_structure}
  \caption{The structure of rough data directory.}
  \label{img:data_dir_structure}
\end{figure}

Right after the data generation, a script called \textit{clean\_txt\_data.py} (\ref{app:code_documentation}) is used to check the created \textit{.txt} files. As it takes a long time to generate all the data, sometimes the simulation fails and the files are incomplete. Hence the script checks whether there are enough timesteps (at least more than 95) and also if the steps are not messed. Files that fail during the inspection are removed.